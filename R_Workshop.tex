\documentclass[]{article}
\usepackage{lmodern}
\usepackage{amssymb,amsmath}
\usepackage{ifxetex,ifluatex}
\usepackage{fixltx2e} % provides \textsubscript
\ifnum 0\ifxetex 1\fi\ifluatex 1\fi=0 % if pdftex
  \usepackage[T1]{fontenc}
  \usepackage[utf8]{inputenc}
\else % if luatex or xelatex
  \ifxetex
    \usepackage{mathspec}
  \else
    \usepackage{fontspec}
  \fi
  \defaultfontfeatures{Ligatures=TeX,Scale=MatchLowercase}
\fi
% use upquote if available, for straight quotes in verbatim environments
\IfFileExists{upquote.sty}{\usepackage{upquote}}{}
% use microtype if available
\IfFileExists{microtype.sty}{%
\usepackage{microtype}
\UseMicrotypeSet[protrusion]{basicmath} % disable protrusion for tt fonts
}{}
\usepackage[margin=1in]{geometry}
\usepackage{hyperref}
\hypersetup{unicode=true,
            pdftitle={R Workshop},
            pdfauthor={Vanessa},
            pdfborder={0 0 0},
            breaklinks=true}
\urlstyle{same}  % don't use monospace font for urls
\usepackage{color}
\usepackage{fancyvrb}
\newcommand{\VerbBar}{|}
\newcommand{\VERB}{\Verb[commandchars=\\\{\}]}
\DefineVerbatimEnvironment{Highlighting}{Verbatim}{commandchars=\\\{\}}
% Add ',fontsize=\small' for more characters per line
\usepackage{framed}
\definecolor{shadecolor}{RGB}{248,248,248}
\newenvironment{Shaded}{\begin{snugshade}}{\end{snugshade}}
\newcommand{\KeywordTok}[1]{\textcolor[rgb]{0.13,0.29,0.53}{\textbf{#1}}}
\newcommand{\DataTypeTok}[1]{\textcolor[rgb]{0.13,0.29,0.53}{#1}}
\newcommand{\DecValTok}[1]{\textcolor[rgb]{0.00,0.00,0.81}{#1}}
\newcommand{\BaseNTok}[1]{\textcolor[rgb]{0.00,0.00,0.81}{#1}}
\newcommand{\FloatTok}[1]{\textcolor[rgb]{0.00,0.00,0.81}{#1}}
\newcommand{\ConstantTok}[1]{\textcolor[rgb]{0.00,0.00,0.00}{#1}}
\newcommand{\CharTok}[1]{\textcolor[rgb]{0.31,0.60,0.02}{#1}}
\newcommand{\SpecialCharTok}[1]{\textcolor[rgb]{0.00,0.00,0.00}{#1}}
\newcommand{\StringTok}[1]{\textcolor[rgb]{0.31,0.60,0.02}{#1}}
\newcommand{\VerbatimStringTok}[1]{\textcolor[rgb]{0.31,0.60,0.02}{#1}}
\newcommand{\SpecialStringTok}[1]{\textcolor[rgb]{0.31,0.60,0.02}{#1}}
\newcommand{\ImportTok}[1]{#1}
\newcommand{\CommentTok}[1]{\textcolor[rgb]{0.56,0.35,0.01}{\textit{#1}}}
\newcommand{\DocumentationTok}[1]{\textcolor[rgb]{0.56,0.35,0.01}{\textbf{\textit{#1}}}}
\newcommand{\AnnotationTok}[1]{\textcolor[rgb]{0.56,0.35,0.01}{\textbf{\textit{#1}}}}
\newcommand{\CommentVarTok}[1]{\textcolor[rgb]{0.56,0.35,0.01}{\textbf{\textit{#1}}}}
\newcommand{\OtherTok}[1]{\textcolor[rgb]{0.56,0.35,0.01}{#1}}
\newcommand{\FunctionTok}[1]{\textcolor[rgb]{0.00,0.00,0.00}{#1}}
\newcommand{\VariableTok}[1]{\textcolor[rgb]{0.00,0.00,0.00}{#1}}
\newcommand{\ControlFlowTok}[1]{\textcolor[rgb]{0.13,0.29,0.53}{\textbf{#1}}}
\newcommand{\OperatorTok}[1]{\textcolor[rgb]{0.81,0.36,0.00}{\textbf{#1}}}
\newcommand{\BuiltInTok}[1]{#1}
\newcommand{\ExtensionTok}[1]{#1}
\newcommand{\PreprocessorTok}[1]{\textcolor[rgb]{0.56,0.35,0.01}{\textit{#1}}}
\newcommand{\AttributeTok}[1]{\textcolor[rgb]{0.77,0.63,0.00}{#1}}
\newcommand{\RegionMarkerTok}[1]{#1}
\newcommand{\InformationTok}[1]{\textcolor[rgb]{0.56,0.35,0.01}{\textbf{\textit{#1}}}}
\newcommand{\WarningTok}[1]{\textcolor[rgb]{0.56,0.35,0.01}{\textbf{\textit{#1}}}}
\newcommand{\AlertTok}[1]{\textcolor[rgb]{0.94,0.16,0.16}{#1}}
\newcommand{\ErrorTok}[1]{\textcolor[rgb]{0.64,0.00,0.00}{\textbf{#1}}}
\newcommand{\NormalTok}[1]{#1}
\usepackage{graphicx,grffile}
\makeatletter
\def\maxwidth{\ifdim\Gin@nat@width>\linewidth\linewidth\else\Gin@nat@width\fi}
\def\maxheight{\ifdim\Gin@nat@height>\textheight\textheight\else\Gin@nat@height\fi}
\makeatother
% Scale images if necessary, so that they will not overflow the page
% margins by default, and it is still possible to overwrite the defaults
% using explicit options in \includegraphics[width, height, ...]{}
\setkeys{Gin}{width=\maxwidth,height=\maxheight,keepaspectratio}
\IfFileExists{parskip.sty}{%
\usepackage{parskip}
}{% else
\setlength{\parindent}{0pt}
\setlength{\parskip}{6pt plus 2pt minus 1pt}
}
\setlength{\emergencystretch}{3em}  % prevent overfull lines
\providecommand{\tightlist}{%
  \setlength{\itemsep}{0pt}\setlength{\parskip}{0pt}}
\setcounter{secnumdepth}{0}
% Redefines (sub)paragraphs to behave more like sections
\ifx\paragraph\undefined\else
\let\oldparagraph\paragraph
\renewcommand{\paragraph}[1]{\oldparagraph{#1}\mbox{}}
\fi
\ifx\subparagraph\undefined\else
\let\oldsubparagraph\subparagraph
\renewcommand{\subparagraph}[1]{\oldsubparagraph{#1}\mbox{}}
\fi

%%% Use protect on footnotes to avoid problems with footnotes in titles
\let\rmarkdownfootnote\footnote%
\def\footnote{\protect\rmarkdownfootnote}

%%% Change title format to be more compact
\usepackage{titling}

% Create subtitle command for use in maketitle
\newcommand{\subtitle}[1]{
  \posttitle{
    \begin{center}\large#1\end{center}
    }
}

\setlength{\droptitle}{-2em}

  \title{R Workshop}
    \pretitle{\vspace{\droptitle}\centering\huge}
  \posttitle{\par}
    \author{Vanessa}
    \preauthor{\centering\large\emph}
  \postauthor{\par}
      \predate{\centering\large\emph}
  \postdate{\par}
    \date{2019-01-21}


\begin{document}
\maketitle

\subsection{Creating Objects in R}\label{creating-objects-in-r}

Super basic, using R as a fancy calculator:

\begin{Shaded}
\begin{Highlighting}[]
\DecValTok{3}\OperatorTok{+}\DecValTok{5}
\end{Highlighting}
\end{Shaded}

\begin{verbatim}
## [1] 8
\end{verbatim}

\begin{Shaded}
\begin{Highlighting}[]
\DecValTok{12}\OperatorTok{/}\DecValTok{7}
\end{Highlighting}
\end{Shaded}

\begin{verbatim}
## [1] 1.714286
\end{verbatim}

\begin{Shaded}
\begin{Highlighting}[]
\DecValTok{5}\OperatorTok{*}\DecValTok{5}
\end{Highlighting}
\end{Shaded}

\begin{verbatim}
## [1] 25
\end{verbatim}

\begin{Shaded}
\begin{Highlighting}[]
\CommentTok{# assign value to object/variable}
\NormalTok{weight_kg <-}\StringTok{ }\DecValTok{55}

\FloatTok{2.2}\OperatorTok{*}\NormalTok{weight_kg}
\end{Highlighting}
\end{Shaded}

\begin{verbatim}
## [1] 121
\end{verbatim}

\begin{Shaded}
\begin{Highlighting}[]
\NormalTok{weight_lb <-}\StringTok{ }\FloatTok{2.2}\OperatorTok{*}\NormalTok{weight_kg}

\KeywordTok{sqrt}\NormalTok{(weight_kg)}
\end{Highlighting}
\end{Shaded}

\begin{verbatim}
## [1] 7.416198
\end{verbatim}

\begin{Shaded}
\begin{Highlighting}[]
\KeywordTok{round}\NormalTok{(pi)}
\end{Highlighting}
\end{Shaded}

\begin{verbatim}
## [1] 3
\end{verbatim}

\begin{Shaded}
\begin{Highlighting}[]
\KeywordTok{round}\NormalTok{(}\FloatTok{3.14159}\NormalTok{)}
\end{Highlighting}
\end{Shaded}

\begin{verbatim}
## [1] 3
\end{verbatim}

\begin{Shaded}
\begin{Highlighting}[]
\KeywordTok{round}\NormalTok{(}\FloatTok{3.14159}\NormalTok{, }\DataTypeTok{digits=}\DecValTok{2}\NormalTok{)}
\end{Highlighting}
\end{Shaded}

\begin{verbatim}
## [1] 3.14
\end{verbatim}

\begin{Shaded}
\begin{Highlighting}[]
\KeywordTok{round}\NormalTok{(pi, }\DataTypeTok{digits =} \DecValTok{6}\NormalTok{)}
\end{Highlighting}
\end{Shaded}

\begin{verbatim}
## [1] 3.141593
\end{verbatim}

\begin{Shaded}
\begin{Highlighting}[]
\KeywordTok{round}\NormalTok{(pi, }\DecValTok{10}\NormalTok{)}
\end{Highlighting}
\end{Shaded}

\begin{verbatim}
## [1] 3.141593
\end{verbatim}

\subsubsection{Vectors and Data Types}\label{vectors-and-data-types}

This section will describe some basic data types in R:

\begin{Shaded}
\begin{Highlighting}[]
\NormalTok{weight_g <-}\StringTok{ }\KeywordTok{c}\NormalTok{(}\DecValTok{50}\NormalTok{, }\DecValTok{60}\NormalTok{, }\DecValTok{65}\NormalTok{, }\DecValTok{82}\NormalTok{)}

\NormalTok{animals <-}\StringTok{ }\KeywordTok{c}\NormalTok{(}\StringTok{"mouse"}\NormalTok{, }\StringTok{"rat"}\NormalTok{, }\StringTok{"dog"}\NormalTok{)}
\end{Highlighting}
\end{Shaded}

Vector types in R:

\begin{itemize}
\tightlist
\item
  numeric
\item
  character
\item
  logical (TRUE or FALSE)
\item
  factors (categorical data i.e.~species)
\item
  Dates
\end{itemize}

A vector is a data structure in R.

Other data structures:

\begin{itemize}
\tightlist
\item
  lists
\item
  data frames
\item
  matrices
\item
  of course vectors
\end{itemize}

Often you want to convert lists and matrices to data frames or vectors.

\subsection{Data Frames}\label{data-frames}

Next we;re going to look at the structure of Data Frames.

\begin{Shaded}
\begin{Highlighting}[]
\KeywordTok{library}\NormalTok{(tidyverse)}
\end{Highlighting}
\end{Shaded}

\begin{verbatim}
## -- Attaching packages -------------------------------------------------- tidyverse 1.2.1 --
\end{verbatim}

\begin{verbatim}
## √ ggplot2 2.2.1     √ purrr   0.2.5
## √ tibble  1.4.2     √ dplyr   0.7.5
## √ tidyr   0.8.1     √ stringr 1.3.1
## √ readr   1.1.1     √ forcats 0.3.0
\end{verbatim}

\begin{verbatim}
## -- Conflicts ----------------------------------------------------- tidyverse_conflicts() --
## x dplyr::filter() masks stats::filter()
## x dplyr::lag()    masks stats::lag()
\end{verbatim}

\begin{Shaded}
\begin{Highlighting}[]
\KeywordTok{download.file}\NormalTok{(}\DataTypeTok{url=}\StringTok{"https://ndownloader.figshare.com/files/2292169"}\NormalTok{, }\DataTypeTok{destfile =} \StringTok{"read_data/portal_data_joined.csv"}\NormalTok{)}

\KeywordTok{library}\NormalTok{(here)}
\end{Highlighting}
\end{Shaded}

\begin{verbatim}
## here() starts at /Users/Vanessa/Desktop/R Projects/Hakai R Workshop
\end{verbatim}

\begin{Shaded}
\begin{Highlighting}[]
\CommentTok{#this package makes working directories and file paths easy}

\NormalTok{surveys <-}\StringTok{ }\KeywordTok{read_csv}\NormalTok{(}\KeywordTok{here}\NormalTok{(}\StringTok{"read_data"}\NormalTok{, }\StringTok{"portal_data_joined.csv"}\NormalTok{))}
\end{Highlighting}
\end{Shaded}

\begin{verbatim}
## Parsed with column specification:
## cols(
##   record_id = col_integer(),
##   month = col_integer(),
##   day = col_integer(),
##   year = col_integer(),
##   plot_id = col_integer(),
##   species_id = col_character(),
##   sex = col_character(),
##   hindfoot_length = col_integer(),
##   weight = col_integer(),
##   genus = col_character(),
##   species = col_character(),
##   taxa = col_character(),
##   plot_type = col_character()
## )
\end{verbatim}

\begin{Shaded}
\begin{Highlighting}[]
\CommentTok{# THIS PART IS NEW AND RELEVANT TO ME }

\KeywordTok{str}\NormalTok{(surveys)}
\end{Highlighting}
\end{Shaded}

\begin{verbatim}
## Classes 'tbl_df', 'tbl' and 'data.frame':    34786 obs. of  13 variables:
##  $ record_id      : int  1 72 224 266 349 363 435 506 588 661 ...
##  $ month          : int  7 8 9 10 11 11 12 1 2 3 ...
##  $ day            : int  16 19 13 16 12 12 10 8 18 11 ...
##  $ year           : int  1977 1977 1977 1977 1977 1977 1977 1978 1978 1978 ...
##  $ plot_id        : int  2 2 2 2 2 2 2 2 2 2 ...
##  $ species_id     : chr  "NL" "NL" "NL" "NL" ...
##  $ sex            : chr  "M" "M" NA NA ...
##  $ hindfoot_length: int  32 31 NA NA NA NA NA NA NA NA ...
##  $ weight         : int  NA NA NA NA NA NA NA NA 218 NA ...
##  $ genus          : chr  "Neotoma" "Neotoma" "Neotoma" "Neotoma" ...
##  $ species        : chr  "albigula" "albigula" "albigula" "albigula" ...
##  $ taxa           : chr  "Rodent" "Rodent" "Rodent" "Rodent" ...
##  $ plot_type      : chr  "Control" "Control" "Control" "Control" ...
##  - attr(*, "spec")=List of 2
##   ..$ cols   :List of 13
##   .. ..$ record_id      : list()
##   .. .. ..- attr(*, "class")= chr  "collector_integer" "collector"
##   .. ..$ month          : list()
##   .. .. ..- attr(*, "class")= chr  "collector_integer" "collector"
##   .. ..$ day            : list()
##   .. .. ..- attr(*, "class")= chr  "collector_integer" "collector"
##   .. ..$ year           : list()
##   .. .. ..- attr(*, "class")= chr  "collector_integer" "collector"
##   .. ..$ plot_id        : list()
##   .. .. ..- attr(*, "class")= chr  "collector_integer" "collector"
##   .. ..$ species_id     : list()
##   .. .. ..- attr(*, "class")= chr  "collector_character" "collector"
##   .. ..$ sex            : list()
##   .. .. ..- attr(*, "class")= chr  "collector_character" "collector"
##   .. ..$ hindfoot_length: list()
##   .. .. ..- attr(*, "class")= chr  "collector_integer" "collector"
##   .. ..$ weight         : list()
##   .. .. ..- attr(*, "class")= chr  "collector_integer" "collector"
##   .. ..$ genus          : list()
##   .. .. ..- attr(*, "class")= chr  "collector_character" "collector"
##   .. ..$ species        : list()
##   .. .. ..- attr(*, "class")= chr  "collector_character" "collector"
##   .. ..$ taxa           : list()
##   .. .. ..- attr(*, "class")= chr  "collector_character" "collector"
##   .. ..$ plot_type      : list()
##   .. .. ..- attr(*, "class")= chr  "collector_character" "collector"
##   ..$ default: list()
##   .. ..- attr(*, "class")= chr  "collector_guess" "collector"
##   ..- attr(*, "class")= chr "col_spec"
\end{verbatim}

\begin{Shaded}
\begin{Highlighting}[]
\KeywordTok{dim}\NormalTok{(surveys)}
\end{Highlighting}
\end{Shaded}

\begin{verbatim}
## [1] 34786    13
\end{verbatim}

\begin{Shaded}
\begin{Highlighting}[]
\KeywordTok{nrow}\NormalTok{(surveys)}
\end{Highlighting}
\end{Shaded}

\begin{verbatim}
## [1] 34786
\end{verbatim}

\begin{Shaded}
\begin{Highlighting}[]
\KeywordTok{ncol}\NormalTok{(surveys)}
\end{Highlighting}
\end{Shaded}

\begin{verbatim}
## [1] 13
\end{verbatim}

\begin{Shaded}
\begin{Highlighting}[]
\KeywordTok{summary}\NormalTok{(surveys)}
\end{Highlighting}
\end{Shaded}

\begin{verbatim}
##    record_id         month             day            year     
##  Min.   :    1   Min.   : 1.000   Min.   : 1.0   Min.   :1977  
##  1st Qu.: 8964   1st Qu.: 4.000   1st Qu.: 9.0   1st Qu.:1984  
##  Median :17762   Median : 6.000   Median :16.0   Median :1990  
##  Mean   :17804   Mean   : 6.474   Mean   :16.1   Mean   :1990  
##  3rd Qu.:26655   3rd Qu.:10.000   3rd Qu.:23.0   3rd Qu.:1997  
##  Max.   :35548   Max.   :12.000   Max.   :31.0   Max.   :2002  
##                                                                
##     plot_id       species_id            sex            hindfoot_length
##  Min.   : 1.00   Length:34786       Length:34786       Min.   : 2.00  
##  1st Qu.: 5.00   Class :character   Class :character   1st Qu.:21.00  
##  Median :11.00   Mode  :character   Mode  :character   Median :32.00  
##  Mean   :11.34                                         Mean   :29.29  
##  3rd Qu.:17.00                                         3rd Qu.:36.00  
##  Max.   :24.00                                         Max.   :70.00  
##                                                        NA's   :3348   
##      weight          genus             species              taxa          
##  Min.   :  4.00   Length:34786       Length:34786       Length:34786      
##  1st Qu.: 20.00   Class :character   Class :character   Class :character  
##  Median : 37.00   Mode  :character   Mode  :character   Mode  :character  
##  Mean   : 42.67                                                           
##  3rd Qu.: 48.00                                                           
##  Max.   :280.00                                                           
##  NA's   :2503                                                             
##   plot_type        
##  Length:34786      
##  Class :character  
##  Mode  :character  
##                    
##                    
##                    
## 
\end{verbatim}

\subsubsection{Indexing and Subsetting Data
Frames}\label{indexing-and-subsetting-data-frames}

First let's use square bracket subsetting.

Square brackets are great for defining coordinates to extract data from.
But what happens when the structure of the data frame changes.

\begin{Shaded}
\begin{Highlighting}[]
\CommentTok{#first define the row coordinate and then the column}
\CommentTok{#also write row and then column}
\NormalTok{surveys[}\DecValTok{1}\NormalTok{, }\DecValTok{1}\NormalTok{]}
\end{Highlighting}
\end{Shaded}

\begin{verbatim}
## # A tibble: 1 x 1
##   record_id
##       <int>
## 1         1
\end{verbatim}

\begin{Shaded}
\begin{Highlighting}[]
\NormalTok{surveys[}\DecValTok{1}\NormalTok{, }\DecValTok{6}\NormalTok{]}
\end{Highlighting}
\end{Shaded}

\begin{verbatim}
## # A tibble: 1 x 1
##   species_id
##   <chr>     
## 1 NL
\end{verbatim}

\begin{Shaded}
\begin{Highlighting}[]
\CommentTok{#defining only which element we want will return a data frame}
\NormalTok{surveys[}\DecValTok{1}\NormalTok{]}
\end{Highlighting}
\end{Shaded}

\begin{verbatim}
## # A tibble: 34,786 x 1
##    record_id
##        <int>
##  1         1
##  2        72
##  3       224
##  4       266
##  5       349
##  6       363
##  7       435
##  8       506
##  9       588
## 10       661
## # ... with 34,776 more rows
\end{verbatim}

\begin{Shaded}
\begin{Highlighting}[]
\NormalTok{surveys[}\DecValTok{1}\OperatorTok{:}\DecValTok{3}\NormalTok{, }\DecValTok{7}\NormalTok{]}
\end{Highlighting}
\end{Shaded}

\begin{verbatim}
## # A tibble: 3 x 1
##   sex  
##   <chr>
## 1 M    
## 2 M    
## 3 <NA>
\end{verbatim}

\begin{Shaded}
\begin{Highlighting}[]
\CommentTok{#give us all the rows and columns except column 7}
\NormalTok{surveys[, }\OperatorTok{-}\DecValTok{7}\NormalTok{]}
\end{Highlighting}
\end{Shaded}

\begin{verbatim}
## # A tibble: 34,786 x 12
##    record_id month   day  year plot_id species_id hindfoot_length weight
##        <int> <int> <int> <int>   <int> <chr>                <int>  <int>
##  1         1     7    16  1977       2 NL                      32     NA
##  2        72     8    19  1977       2 NL                      31     NA
##  3       224     9    13  1977       2 NL                      NA     NA
##  4       266    10    16  1977       2 NL                      NA     NA
##  5       349    11    12  1977       2 NL                      NA     NA
##  6       363    11    12  1977       2 NL                      NA     NA
##  7       435    12    10  1977       2 NL                      NA     NA
##  8       506     1     8  1978       2 NL                      NA     NA
##  9       588     2    18  1978       2 NL                      NA    218
## 10       661     3    11  1978       2 NL                      NA     NA
## # ... with 34,776 more rows, and 4 more variables: genus <chr>,
## #   species <chr>, taxa <chr>, plot_type <chr>
\end{verbatim}

\begin{Shaded}
\begin{Highlighting}[]
\NormalTok{surveys[, }\OperatorTok{-}\KeywordTok{c}\NormalTok{(}\DecValTok{1}\OperatorTok{:}\DecValTok{3}\NormalTok{)]}
\end{Highlighting}
\end{Shaded}

\begin{verbatim}
## # A tibble: 34,786 x 10
##     year plot_id species_id sex   hindfoot_length weight genus   species 
##    <int>   <int> <chr>      <chr>           <int>  <int> <chr>   <chr>   
##  1  1977       2 NL         M                  32     NA Neotoma albigula
##  2  1977       2 NL         M                  31     NA Neotoma albigula
##  3  1977       2 NL         <NA>               NA     NA Neotoma albigula
##  4  1977       2 NL         <NA>               NA     NA Neotoma albigula
##  5  1977       2 NL         <NA>               NA     NA Neotoma albigula
##  6  1977       2 NL         <NA>               NA     NA Neotoma albigula
##  7  1977       2 NL         <NA>               NA     NA Neotoma albigula
##  8  1978       2 NL         <NA>               NA     NA Neotoma albigula
##  9  1978       2 NL         M                  NA    218 Neotoma albigula
## 10  1978       2 NL         <NA>               NA     NA Neotoma albigula
## # ... with 34,776 more rows, and 2 more variables: taxa <chr>,
## #   plot_type <chr>
\end{verbatim}

\subsection{Data Manipulation}\label{data-manipulation}

Key functions for data manipulation:

\begin{itemize}
\tightlist
\item
  \texttt{select()}: subsetting columns
\item
  \texttt{filter()}: subsets of rows based on conditions
\item
  \texttt{mutate()}: create new columns, based on information from other
  columns
\item
  \texttt{group\_by()}: creates groups based on categorical data in a
  column
\item
  \texttt{summarize()}: creates summary stats on grouped data
\item
  \texttt{arrange()}: sort results
\item
  \texttt{count()}: gives a count of discrete values
\end{itemize}

\begin{Shaded}
\begin{Highlighting}[]
\KeywordTok{select}\NormalTok{(surveys, plot_id, species_id, weight)}
\end{Highlighting}
\end{Shaded}

\begin{verbatim}
## # A tibble: 34,786 x 3
##    plot_id species_id weight
##      <int> <chr>       <int>
##  1       2 NL             NA
##  2       2 NL             NA
##  3       2 NL             NA
##  4       2 NL             NA
##  5       2 NL             NA
##  6       2 NL             NA
##  7       2 NL             NA
##  8       2 NL             NA
##  9       2 NL            218
## 10       2 NL             NA
## # ... with 34,776 more rows
\end{verbatim}

\begin{Shaded}
\begin{Highlighting}[]
\CommentTok{#negative subsetting}
\KeywordTok{select}\NormalTok{(surveys, }\OperatorTok{-}\NormalTok{record_id)}
\end{Highlighting}
\end{Shaded}

\begin{verbatim}
## # A tibble: 34,786 x 12
##    month   day  year plot_id species_id sex   hindfoot_length weight genus
##    <int> <int> <int>   <int> <chr>      <chr>           <int>  <int> <chr>
##  1     7    16  1977       2 NL         M                  32     NA Neot~
##  2     8    19  1977       2 NL         M                  31     NA Neot~
##  3     9    13  1977       2 NL         <NA>               NA     NA Neot~
##  4    10    16  1977       2 NL         <NA>               NA     NA Neot~
##  5    11    12  1977       2 NL         <NA>               NA     NA Neot~
##  6    11    12  1977       2 NL         <NA>               NA     NA Neot~
##  7    12    10  1977       2 NL         <NA>               NA     NA Neot~
##  8     1     8  1978       2 NL         <NA>               NA     NA Neot~
##  9     2    18  1978       2 NL         M                  NA    218 Neot~
## 10     3    11  1978       2 NL         <NA>               NA     NA Neot~
## # ... with 34,776 more rows, and 3 more variables: species <chr>,
## #   taxa <chr>, plot_type <chr>
\end{verbatim}

\begin{Shaded}
\begin{Highlighting}[]
\KeywordTok{filter}\NormalTok{(surveys, year}\OperatorTok{==}\DecValTok{1995}\NormalTok{,}
\NormalTok{       species_id}\OperatorTok{==}\StringTok{"NL"}\NormalTok{)}
\end{Highlighting}
\end{Shaded}

\begin{verbatim}
## # A tibble: 8 x 13
##   record_id month   day  year plot_id species_id sex   hindfoot_length
##       <int> <int> <int> <int>   <int> <chr>      <chr>           <int>
## 1     22314     6     7  1995       2 NL         M                  34
## 2     22728     9    23  1995       2 NL         F                  32
## 3     22899    10    28  1995       2 NL         F                  32
## 4     23032    12     2  1995       2 NL         F                  33
## 5     22847    10    28  1995      12 NL         M                  34
## 6     22998    12     2  1995      12 NL         M                  33
## 7     23124    12    21  1995      12 NL         F                  32
## 8     22476     7    20  1995      24 NL         F                  31
## # ... with 5 more variables: weight <int>, genus <chr>, species <chr>,
## #   taxa <chr>, plot_type <chr>
\end{verbatim}

\subsection{Pipes}\label{pipes}

Pipes allow you to chain together dplyr functions.

Pipe: \%\textgreater{}\% or cmd-shift-m

\begin{Shaded}
\begin{Highlighting}[]
\CommentTok{#write multiple arguments in a sentence using pipes}
\NormalTok{surveys }\OperatorTok
\StringTok{  }\KeywordTok{filter}\NormalTok{(weight}\OperatorTok{<}\DecValTok{5}\NormalTok{) }\OperatorTok
\StringTok{  }\KeywordTok{select}\NormalTok{(species_id, sex, weight)}
\end{Highlighting}
\end{Shaded}

\begin{verbatim}
## # A tibble: 17 x 3
##    species_id sex   weight
##    <chr>      <chr>  <int>
##  1 PF         F          4
##  2 PF         F          4
##  3 PF         M          4
##  4 RM         F          4
##  5 RM         M          4
##  6 PF         <NA>       4
##  7 PP         M          4
##  8 RM         M          4
##  9 RM         M          4
## 10 RM         M          4
## 11 PF         M          4
## 12 PF         F          4
## 13 RM         M          4
## 14 RM         M          4
## 15 RM         F          4
## 16 RM         M          4
## 17 RM         M          4
\end{verbatim}

\begin{Shaded}
\begin{Highlighting}[]
\NormalTok{surveys_sml <-}\StringTok{ }\NormalTok{surveys }\OperatorTok
\StringTok{  }\KeywordTok{filter}\NormalTok{(weight}\OperatorTok{<}\DecValTok{5}\NormalTok{) }\OperatorTok
\StringTok{  }\KeywordTok{select}\NormalTok{(species_id, sex, weight)}
\end{Highlighting}
\end{Shaded}

Challenge \#1

Using pipe, subset the surveys dataframe to include animals collected
1995 and retain only the columns year, sex and weight.

\begin{Shaded}
\begin{Highlighting}[]
\NormalTok{surveys }\OperatorTok
\StringTok{  }\KeywordTok{filter}\NormalTok{(year}\OperatorTok{==}\DecValTok{1995}\NormalTok{) }\OperatorTok
\StringTok{  }\KeywordTok{select}\NormalTok{(year, sex, weight)}
\end{Highlighting}
\end{Shaded}

\begin{verbatim}
## # A tibble: 1,180 x 3
##     year sex   weight
##    <int> <chr>  <int>
##  1  1995 M         NA
##  2  1995 F        165
##  3  1995 F        171
##  4  1995 F         NA
##  5  1995 M         41
##  6  1995 F         45
##  7  1995 M         46
##  8  1995 F         49
##  9  1995 M         46
## 10  1995 M         48
## # ... with 1,170 more rows
\end{verbatim}

\begin{Shaded}
\begin{Highlighting}[]
\NormalTok{surveys }\OperatorTok
\StringTok{  }\KeywordTok{mutate}\NormalTok{(}\DataTypeTok{weight_kg=}\NormalTok{weight}\OperatorTok{/}\DecValTok{1000}\NormalTok{,}
         \DataTypeTok{weight_kg2=}\NormalTok{weight_kg}\OperatorTok{*}\DecValTok{2}\NormalTok{)}
\end{Highlighting}
\end{Shaded}

\begin{verbatim}
## # A tibble: 34,786 x 15
##    record_id month   day  year plot_id species_id sex   hindfoot_length
##        <int> <int> <int> <int>   <int> <chr>      <chr>           <int>
##  1         1     7    16  1977       2 NL         M                  32
##  2        72     8    19  1977       2 NL         M                  31
##  3       224     9    13  1977       2 NL         <NA>               NA
##  4       266    10    16  1977       2 NL         <NA>               NA
##  5       349    11    12  1977       2 NL         <NA>               NA
##  6       363    11    12  1977       2 NL         <NA>               NA
##  7       435    12    10  1977       2 NL         <NA>               NA
##  8       506     1     8  1978       2 NL         <NA>               NA
##  9       588     2    18  1978       2 NL         M                  NA
## 10       661     3    11  1978       2 NL         <NA>               NA
## # ... with 34,776 more rows, and 7 more variables: weight <int>,
## #   genus <chr>, species <chr>, taxa <chr>, plot_type <chr>,
## #   weight_kg <dbl>, weight_kg2 <dbl>
\end{verbatim}

\begin{Shaded}
\begin{Highlighting}[]
\NormalTok{surveys <-}\StringTok{ }\NormalTok{surveys }\OperatorTok
\StringTok{  }\KeywordTok{drop_na}\NormalTok{(weight) }\OperatorTok
\StringTok{  }\KeywordTok{mutate}\NormalTok{(}\DataTypeTok{mean_weight=}\KeywordTok{mean}\NormalTok{(weight))}
\end{Highlighting}
\end{Shaded}

Challenge \#2

Using the surveys data from create a new data frame that contains only
the species\_id column, has a new column called hindfoot\_half: contains
values that are half the hindfoot\_length values. Also, in the new
hindfoot\_half column there are no NAs and values are all less than 30.

\begin{Shaded}
\begin{Highlighting}[]
\NormalTok{surveys_hindfoot_half <-}\StringTok{ }\NormalTok{surveys }\OperatorTok
\StringTok{  }\KeywordTok{drop_na}\NormalTok{(hindfoot_length) }\OperatorTok\StringTok{ }
\StringTok{  }\KeywordTok{mutate}\NormalTok{(}\DataTypeTok{hindfoot_half=}\NormalTok{hindfoot_length}\OperatorTok{/}\DecValTok{2}\NormalTok{) }\OperatorTok
\StringTok{  }\KeywordTok{filter}\NormalTok{(hindfoot_half}\OperatorTok{<}\DecValTok{30}\NormalTok{) }\OperatorTok\StringTok{ }
\StringTok{  }\KeywordTok{select}\NormalTok{(species_id, hindfoot_half, hindfoot_length)}
\end{Highlighting}
\end{Shaded}

\begin{Shaded}
\begin{Highlighting}[]
\NormalTok{surveys }\OperatorTok
\StringTok{  }\KeywordTok{group_by}\NormalTok{(sex) }\OperatorTok
\StringTok{  }\KeywordTok{summarize}\NormalTok{(}\DataTypeTok{mean_weight=}\KeywordTok{mean}\NormalTok{(weight, }\DataTypeTok{na.rm =} \OtherTok{TRUE}\NormalTok{))}
\end{Highlighting}
\end{Shaded}

\begin{verbatim}
## # A tibble: 3 x 2
##   sex   mean_weight
##   <chr>       <dbl>
## 1 F            42.2
## 2 M            43.0
## 3 <NA>         64.7
\end{verbatim}

\begin{Shaded}
\begin{Highlighting}[]
\NormalTok{surveys }\OperatorTok
\StringTok{  }\KeywordTok{group_by}\NormalTok{(sex, species_id) }\OperatorTok
\StringTok{  }\KeywordTok{summarize}\NormalTok{(}\DataTypeTok{mean_weight=}\KeywordTok{mean}\NormalTok{(weight, }\DataTypeTok{na.rm=}\OtherTok{TRUE}\NormalTok{),}
            \DataTypeTok{min_weight=}\KeywordTok{min}\NormalTok{(weight, }\DataTypeTok{na.rm=}\OtherTok{TRUE}\NormalTok{)) }\OperatorTok
\StringTok{  }\KeywordTok{arrange}\NormalTok{(}\KeywordTok{desc}\NormalTok{(min_weight))}
\end{Highlighting}
\end{Shaded}

\begin{verbatim}
## # A tibble: 64 x 4
## # Groups:   sex [3]
##    sex   species_id mean_weight min_weight
##    <chr> <chr>            <dbl>      <dbl>
##  1 M     SS               130          130
##  2 <NA>  SH               130          130
##  3 <NA>  NL               168.          83
##  4 <NA>  DS               120           78
##  5 F     SS                57           57
##  6 F     SF                69           46
##  7 F     DS               118.          45
##  8 <NA>  DO                50.7         44
##  9 <NA>  SF                40.5         36
## 10 M     SO                55.7         35
## # ... with 54 more rows
\end{verbatim}

\begin{Shaded}
\begin{Highlighting}[]
\NormalTok{surveys }\OperatorTok
\StringTok{  }\KeywordTok{count}\NormalTok{(sex, }\DataTypeTok{sort =} \OtherTok{TRUE}\NormalTok{)}
\end{Highlighting}
\end{Shaded}

\begin{verbatim}
## # A tibble: 3 x 2
##   sex       n
##   <chr> <int>
## 1 M     16879
## 2 F     15303
## 3 <NA>    101
\end{verbatim}

\begin{Shaded}
\begin{Highlighting}[]
\CommentTok{#the above code is synonomous with}
\NormalTok{surveys }\OperatorTok\StringTok{ }
\StringTok{  }\KeywordTok{group_by}\NormalTok{(sex) }\OperatorTok
\StringTok{  }\KeywordTok{summarise}\NormalTok{(}\DataTypeTok{count=}\KeywordTok{n}\NormalTok{())}
\end{Highlighting}
\end{Shaded}

\begin{verbatim}
## # A tibble: 3 x 2
##   sex   count
##   <chr> <int>
## 1 F     15303
## 2 M     16879
## 3 <NA>    101
\end{verbatim}

Challenge \# 3

How many animals were caught in each plot\_type surveyed.

\begin{Shaded}
\begin{Highlighting}[]
\NormalTok{surveys }\OperatorTok
\StringTok{  }\KeywordTok{count}\NormalTok{(plot_type)}
\end{Highlighting}
\end{Shaded}

\begin{verbatim}
## # A tibble: 5 x 2
##   plot_type                     n
##   <chr>                     <int>
## 1 Control                   14652
## 2 Long-term Krat Exclosure   4692
## 3 Rodent Exclosure           3818
## 4 Short-term Krat Exclosure  5407
## 5 Spectab exclosure          3714
\end{verbatim}

Use group\_by and summarize to find the mean, min and max of hindfoot
length (using species\_id) for each species. Also, add the number of
observations (hint: see ?n)

\begin{Shaded}
\begin{Highlighting}[]
\NormalTok{surveys }\OperatorTok
\StringTok{  }\KeywordTok{group_by}\NormalTok{(species_id) }\OperatorTok
\StringTok{  }\KeywordTok{summarise}\NormalTok{(}\DataTypeTok{mean_length=}\KeywordTok{mean}\NormalTok{(hindfoot_length, }\DataTypeTok{na.rm =} \OtherTok{TRUE}\NormalTok{),}
            \DataTypeTok{min_length=}\KeywordTok{min}\NormalTok{(hindfoot_length, }\DataTypeTok{na.rm =} \OtherTok{TRUE}\NormalTok{),}
            \DataTypeTok{max_length=}\KeywordTok{max}\NormalTok{(hindfoot_length, }\DataTypeTok{na.rm =} \OtherTok{TRUE}\NormalTok{), }\DataTypeTok{n=}\KeywordTok{n}\NormalTok{())}
\end{Highlighting}
\end{Shaded}

\begin{verbatim}
## # A tibble: 25 x 5
##    species_id mean_length min_length max_length     n
##    <chr>            <dbl>      <dbl>      <dbl> <int>
##  1 BA                13            6         16    45
##  2 DM                36.0         16         50 10262
##  3 DO                35.6         26         64  2904
##  4 DS                50.0         39         58  2344
##  5 NL                32.3         21         42  1152
##  6 OL                20.5         12         39   970
##  7 OT                20.3         13         50  2160
##  8 OX                20.4         19         21     6
##  9 PB                26.1          2         47  2810
## 10 PE                20.2         11         30  1260
## # ... with 15 more rows
\end{verbatim}

What was the heaviest animal measured in each year? Return the columns
year, genus, species\_id and weight.

\begin{Shaded}
\begin{Highlighting}[]
\NormalTok{surveys }\OperatorTok
\StringTok{  }\KeywordTok{group_by}\NormalTok{(year) }\OperatorTok
\StringTok{  }\KeywordTok{summarise}\NormalTok{(}\DataTypeTok{genus=}\KeywordTok{first}\NormalTok{(genus),}
            \DataTypeTok{species_id=}\KeywordTok{first}\NormalTok{(species_id),}
    \DataTypeTok{max_weight=}\KeywordTok{max}\NormalTok{(weight, }\DataTypeTok{na.rm=}\OtherTok{TRUE}\NormalTok{)) }\OperatorTok\StringTok{ }
\StringTok{  }\KeywordTok{select}\NormalTok{(year, genus, species_id, max_weight)}
\end{Highlighting}
\end{Shaded}

\begin{verbatim}
## # A tibble: 26 x 4
##     year genus     species_id max_weight
##    <int> <chr>     <chr>           <dbl>
##  1  1977 Dipodomys DM                149
##  2  1978 Neotoma   NL                232
##  3  1979 Neotoma   NL                274
##  4  1980 Neotoma   NL                243
##  5  1981 Neotoma   NL                264
##  6  1982 Neotoma   NL                252
##  7  1983 Neotoma   NL                256
##  8  1984 Neotoma   NL                259
##  9  1985 Neotoma   NL                225
## 10  1986 Neotoma   NL                240
## # ... with 16 more rows
\end{verbatim}

\begin{Shaded}
\begin{Highlighting}[]
\CommentTok{#my version}
\CommentTok{#incorrect! grabbed the first name and filled in the relevant value}
\CommentTok{#all neotoma except 1977 where it displayed dipodomys because no neot.}

\NormalTok{max_weights <-}\StringTok{ }\NormalTok{surveys }\OperatorTok
\StringTok{  }\KeywordTok{drop_na}\NormalTok{(weight) }\OperatorTok
\StringTok{  }\KeywordTok{group_by}\NormalTok{(year) }\OperatorTok
\StringTok{  }\KeywordTok{filter}\NormalTok{(weight}\OperatorTok{==}\KeywordTok{max}\NormalTok{(weight)) }\OperatorTok
\StringTok{  }\KeywordTok{select}\NormalTok{(year, genus, species_id, weight) }\OperatorTok
\StringTok{  }\KeywordTok{arrange}\NormalTok{(year) }\OperatorTok
\StringTok{  }\KeywordTok{unique}\NormalTok{()}
\CommentTok{#brett's version - correct way to do it! wouldn't thought of filter}
\end{Highlighting}
\end{Shaded}

\subsection{Export Our Data}\label{export-our-data}

\begin{Shaded}
\begin{Highlighting}[]
\KeywordTok{write_csv}\NormalTok{(max_weights, }\KeywordTok{here}\NormalTok{(}\StringTok{"write_data"}\NormalTok{, }\StringTok{"max_weights.csv"}\NormalTok{))}
\end{Highlighting}
\end{Shaded}

Git history is stored locally here. Hit diff/history for more details.


\end{document}
